%@TheDoctorRAB
%standard white paper/preproposal format
%
%%%%%
%
%REFERENCES
%
%neup.bst - numbered citations in order of appearance, short author list with et al in reference section
%nsf.bst - numbered citations in order of appearance, full author list in references section
%standard.bst - citations with author last name with et al for more than 2 authors; full author list in references section
%ans.bst is for ANS only. 
%
%author = {Lastname, Firstname and Lastname, Firstname and Lastname, Firstname} for all bst formats
%bst renders the author list itself
%
%author = {{Nuclear Regulatory Commission}} if the author is an organization, institution, etc., and not people
%
%title = {{}} for all
%
%for all - use \citep{-} - [1] or (Borrelli, 2021) in the text
%standard.bst \cite{-} - Borrelli (2021) in the text
%standard.bst lists references alphabetically
%the rest list numerically
%
%%%%%
\documentclass[11pt,a4paper]{article}
\usepackage[lmargin=1in,rmargin=1in,tmargin=1in,bmargin=1in]{geometry}
\usepackage[pagewise,modulo]{lineno} %line numbering
\usepackage{setspace}
\usepackage{ulem} %strikethrough - do not \sout{\cite{}}
\usepackage[pdftex,dvipsnames]{xcolor,colortbl} %change font color
\usepackage{graphicx}
%\usepackage{filecontents}
\usepackage{tablefootnote}
\usepackage{footnotehyper}
\usepackage{float}
%\usepackage{subfig}
\usepackage[yyyymmdd]{datetime} %date format
\renewcommand{\dateseparator}{.}
\graphicspath{{../img/}} %path to graphics
\setcounter{secnumdepth}{5} %set subsection to nth level

%%%%% fonts
\usepackage{times}
%arial - uncomment next two lines
%\usepackage{helvet}
%\renewcommand{\familydefault}{\sfdefault}
%%%%%

%%%%% references
%\usepackage[round,semicolon]{natbib} %for (Borrelli 2021; Clooney 2019) - standard.bst 
\usepackage[numbers,sort&compress]{natbib} %for [1-3] - nsf.bst, neup.bst
%\setlength{\bibsep}{7pt} %sets space between references
%\renewcommand{\bibsection}{} %suppresses large 'references' heading
%\renewcommand\bibpreamble{\vspace{\baselineskip}} %sets spacing after heading if not using default references heading
%%%%%

%%%%% tables and figures
\usepackage{longtable} %need to put label at top under caption then \\ - use spacing
\usepackage{tablefootnote}
\usepackage{tabularx}
\usepackage{multirow}
\usepackage{tabto} %general tabbed spacing
\usepackage{pdfpages}

\usepackage{wrapfig} %wraps figures around text
\setlength{\intextsep}{0.05mm}
\setlength{\columnsep}{0.05mm}

\usepackage[singlelinecheck=false,labelfont=bf]{caption}
\usepackage{subcaption}
\captionsetup[table]{justification=justified,skip=7pt,labelformat={default},labelsep=period,name={Table}} %sets a space after table caption
\captionsetup[figure]{justification=justified,skip=7pt,labelformat={default},labelsep=period,name={Figure}} %sets space above caption, 'figure' format

%\captionsetup[wrapfigure]{justification=centering,aboveskip=0pt,belowskip=1pt,labelformat={default},labelsep=period,name={Fig.}} %sets space above caption, 'figure' format
%\captionsetup[wraptable]{justification=centering,aboveskip=0pt,belowskip=0pt,labelformat={default},labelsep=period,name={Tab.}} %sets space above caption, 'figure' format
%%%%%

\usepackage[stable,hang,flushmargin]{footmisc} %footnotes in section titles and no indent; standard.bst
\usepackage[inline]{enumitem}
\usepackage{boldline}
\usepackage{makecell}
\usepackage{booktabs}
\usepackage{amssymb}
\usepackage{gensymb}
\usepackage{amsmath,nicefrac}
\usepackage{physics}
\usepackage{lscape}
\usepackage{array}
\usepackage{chngcntr}
\usepackage[hidelinks]{hyperref}
\usepackage{sectsty}
\usepackage{textcomp}
\usepackage{lastpage}
\usepackage{xargs} %for \newcommandx
\usepackage[colorinlistoftodos,prependcaption,textsize=tiny]{todonotes} %makes colored boxes for commenting
\usepackage{marginnote}
\usepackage{gensymb}
\usepackage[colorinlistoftodos,prependcaption,textsize=tiny]{todonotes} %makes colored boxes for commenting
\usepackage[toc,title]{appendix}
\usepackage[figure,table]{totalcount}

\usepackage[capitalise]{cleveref}

%%%%% watermark
\usepackage[firstpage,vpos=0.67\paperheight]{draftwatermark}
\SetWatermarkText{\shortstack{FIRST DRAFT\\do not distribute}}
\SetWatermarkScale{0.20}
\SetWatermarkScale{0.25}
%%%%% 

%\usepackage{xr} %for revisions - will cross reference from one file to here
%\externaldocument{/path/to/auxfilename} %aux file needed

%%%%% toc and glossaries
\usepackage[acronym,nomain,nonumberlist]{glossaries}
\makenoidxglossaries
\usepackage{titlesec,titletoc}
%\renewcommand{\thepart}{ARTICLE \Roman{part}} %puts the label into the command so \thelabel will carry through
%\renewcommand{\thesection}{\arabic{section}} %puts the label into the command so \thelabel will carry through
%\titleformat{\part}{\normalfont\Large\bfseries\filcenter}{\thepart}{}{}[]
%\titlespacing\part{0pt}{0.75\baselineskip}{0.50\baselineskip}
%\titleformat{\section}[runin]{\normalfont\large\bfseries}{\thesection}{-1em}{}[.]
%\titlespacing*\section{0pt}{0.55\baselineskip}{0.35\baselineskip}
%\titleformat{\subsection}[runin]{\normalfont\normalsize\bfseries}{\thesubsection}{-1em}{}[.]
%\titlespacing*\subsection{0pt}{0.45\baselineskip}{0.30\baselineskip}
\titleformat{\paragraph}[runin]{\normalfont\normalsize\bfseries\itshape}{\theparagraph}{-1em}{}[.]
\titlespacing*\paragraph{0pt}{0.50\baselineskip}{0.25\baselineskip}
\titleformat{\subparagraph}[runin]{\normalfont\normalsize\itshape}{\thesubparagraph}{-1em}{}[.]
\titlespacing*\subparagraph{0pt}{0.40\baselineskip}{0.20\baselineskip}

\newcommand{\edit}[1]{\textcolor{blue}{#1}} %shortcut for changing font color on revised text
\newcommand{\fn}[1]{\footnote{#1}} %shortcut for footnote tag
\newcommand*\sq{\mathbin{\vcenter{\hbox{\rule{.3ex}{.3ex}}}}} %makes a small square as a separator $\sq$
\newcommand{\sk}[1]{\sout{#1}} %shortcut for strikethrough
\newcommand{\x}{\cellcolor{lightgray}} %use to shade in table cell

\newcommand{\acf}{\acrfull} %full acronym
\newcommand{\acl}{\acrlong} %long acronym
\newcommand{\acs}{\acrshort} %short acronym

\newcommand{\acfp}{\acrfullpl} %full acronym plural
\newcommand{\aclp}{\acrlongpl} %long acronym plural
\newcommand{\acsp}{\acrshortpl} %short acronym plural

\newacronym{msr}{MSR}{Molten Salt Reactor}
\newacronym{msre}{MSRE}{Molten Salt Reactor Experiment}
\newacronym{smr}{SMR}{Small Modular Reactor}
\newacronym{oak}{ORNL}{Oak Ridge National Laboratory}
\newacronym{msnb}{MSNB}{Molten Salt Nuclear Battery}
\newacronym{haleu}{HALEU}{High-Assay Low-Enrichment Uranium}
\newacronym{nrc}{NRC}{Nuclear Regulatory Comission}
\newacronym{hpc}{HPC}{High Performance Computing}
\newacronym{inl}{INL}{Idaho National Laboratory}
\newacronym{doe}{DOE}{Department of Energy}
\newacronym{pid}{PID}{Proportional-Integral-Derivative}
\newacronym{nsuf}{NSUF}{Nuclear Science User Facility}

%Nuclides
\newcommand{\I}[1][135]{$^{135}I$ }
\newcommand{\Xe}[1][135]{$^{#1}Xe$ }
\newcommand{\U}[1][]{$^{#1}U$ }

%Nomeclature


\newcommandx{\que}[2][1=]{\todo[linecolor=red,backgroundcolor=red!25,bordercolor=red,#1]{#2}} %query
\newcommandx{\por}[2][1=]{\todo[author=Porter,linecolor=blue,backgroundcolor=blue!25,bordercolor=blue,#1]{#2}} %suggested change
%\newcommandx{\}[2][1=]{\todo[author=Sam,linecolor=OliveGreen,backgroundcolor=OliveGreen!25,bordercolor=OliveGreen,#1]{#2}} %comment
%\newcommandx{\}[2][1=]{\todo[author=RAB,tickmarkheight=0.15cm,linecolor=Plum,backgroundcolor=Plum!25,bordercolor=Plum,#1]{#2}} %omit
\newcommandx{\sug}[2][1=]{\todo[linecolor=blue,backgroundcolor=blue!25,bordercolor=blue,#1]{#2}} %suggested change
\newcommandx{\cmt}[2][1=]{\todo[linecolor=OliveGreen,backgroundcolor=OliveGreen!25,bordercolor=OliveGreen,#1]{#2}} %comment
\newcommandx{\rab}[2][1=]{\author=RAB,tickmarkheight=0.15cm,todo[linecolor=Plum,backgroundcolor=Plum!25,bordercolor=Plum,#1]{#2}} %omit

\newcolumntype{L}[1]{>{\raggedright\let\newline\\\arraybackslash\hspace{0pt}}m{#1}} %uses \raggedright with m,p{} in table column
\newcolumntype{C}[1]{>{\centering\let\newline\\\arraybackslash\hspace{0pt}}m{#1}} %uses \raggedright with m,p{} in table column
\newcolumntype{R}[1]{>{\raggedleft\let\newline\\\arraybackslash\hspace{0pt}}m{#1}} %uses \raggedright with m,p{} in table column

\makeatletter
\renewcommand\tableofcontents{%
    \@starttoc{toc}%
}
\makeatother

\makeatletter
\renewcommand\listoffigures{%
    \@starttoc{lof}%
}
\makeatother

\makeatletter
\renewcommand\listoftables{%
    \@starttoc{lot}%
}
\makeatother

\makeatletter
\newcommand*\ftp{\fontsize{16.5}{17.5}\selectfont}
\makeatother

%\makeatletter
%\renewcommand\section{%
%    \@startsection{section}{1}{\z@ }{0.50\baselineskip}{0.25\baselineskip}
%    {\large \normalfont \bfseries}}%

%\makeatletter
%\renewcommand\paragraph{%
%    \@startsection{paragraph}{4}{\z@ }{0.55\baselineskip}{-1em}
%    {\normalfont \normalsize \bfseries}}%

%\makeatletter
%\renewcommand\subparagraph{%
%    \@startsection{subparagraph}{5}{\z@ }{0.40\baselineskip}{-1em}
%    {\normalfont \normalsize \itshape }}%

%\makeatletter
%\renewcommand\subsection{%
%    \@startsection{subsection}{2}{\z@ }{0.45\baselineskip}{0.25\baselineskip}
%    {\large \normalfont \bfseries}}%
    
%%%%% header and footer
\usepackage{fancyhdr}
\pagestyle{fancy}
\fancyhf{} %move page number to bottom right
%\renewcommand{\headrulewidth}{0pt} %set line thickness in header; uncomment as is to remove line
\lhead{\scriptsize \acs{msr} Modeling and Control}
\chead{\scriptsize \textit{Root et al. - Draft Manuscript}}
\rhead{\scriptsize \today}
\rfoot{\thepage}
%%%%%

%Make post-it notes!
\usepackage[colorinlistoftodos,prependcaption,textsize=tiny]{todonotes}
\newcommandx{\note}[2][1=]{\todo[linecolor=orange,backgroundcolor=yellow!25,bordercolor=orange,#1]{#2}}

\begin{document}


\begin{titlepage}
    \title{Dynamic System Modeling and Controller Design for a Molten Salt Microreactor}
    \author{
        \textsuperscript{a,*}Sam J. Root,
        \textsuperscript{a}R. A. Borrelli,
        \textsuperscript{b}Dakota Roberson,
        \textsuperscript{a}Michael G. McKellar
        \\ \\ \\
        University of Idaho $\sq$ Idaho Falls Center for Higher Education\\
        \textsuperscript{a}Department of Nuclear Engineering \& Industrial Management\\
        \textsuperscript{b}Department of Electrical \& Computer Engineering\\
        \\ \\ \\
        \textsuperscript{*}root2892@vandals.uidaho.edu
    }
\clearpage %not have page number on title page
\maketitle
\vspace*{\fill}
\begin{flushright}{
        \noindent Number of pages - \pageref{LastPage} \\
        \noindent Number of tables - \totaltables \\
        \noindent Number of figures - \totalfigures 
}
\end{flushright}
\thispagestyle{empty} %start with page number 1 on second page
\end{titlepage}


%\listoftodos[List of revisions]
%\newpage


%%%%% spacing
\onehalfspacing %linespacing
%\setstretch{1.05} %linespacing
%\spacing{1.25} %equivalent to 1.5 line spacing in Word
%%%%%


%%%%% linenumbering
\linenumbers %toggle line numbers
\pagewiselinenumbers %reset line numbers on new page
\modulolinenumbers[3] %line numbering interval
%%%%%


\section*{Abstract}


\newpage


\printnoidxglossary


\newpage
%\section*{Table of Contents}
%\tableofcontents
%\newpage


\section{Introduction} \label{sec-intro}
\cite{RootThesis}
\subsection{Motivation} \label{sec-motiv}

\subsection{Goals} \label{sec-goals}
We ask the following questions - 
\begin{enumerate}[topsep=0pt,itemsep=-0.75ex,partopsep=1ex,parsep=1ex,leftmargin=*,label=(\arabic*)]
    \item 1 
    \item 2
    \item 3
\end{enumerate}

To address these, we have the following goals for this paper - 
\begin{enumerate}[topsep=0pt,itemsep=-0.75ex,partopsep=1ex,parsep=1ex,leftmargin=*,label=(\arabic*)]
    \item a
    \item b
    \item c
\end{enumerate}


\newpage


\section{Background} \label{sec-bak}


\newpage


\section{Theory} \label{sec-theo}


\newpage


\section{Methodology} \label{sec-meth}


\newpage


\section{Results} \label{sec-res}
\subsection{Objective} \label{sec-obj}


\newpage


\section{Discussion} \label{sec-disc}


\newpage


\section{Future work} \label{sec-fwk}


\newpage


\section{Summary remarks} \label{sec-sum}

Major results and implications include - 
\begin{itemize}[topsep=0pt,itemsep=-0.75ex,partopsep=1ex,parsep=1ex,leftmargin=*]
    \item 1
    \item 2
    \item 3
\end{itemize}


\newpage


\section*{Acknowledgements}
This work was completed under a Graduate Fellowship funded by \acf{nrc}.
   
This research made use of the resources of the High Performance Computing Center at Idaho National Laboratory, which is supported by the Office of Nuclear Energy of the U.S. Department of Energy and the Nuclear Science User Facilities under Contract No. DE-AC07-05ID14517.

\newpage


\bibliographystyle{./rcs/standard}
\setlength{\bibhang}{0pt}
\bibliography{./rcs/bibliography}


\newpage


\section*{Tables}
{%
\let\oldnumberline\numberline%
\renewcommand{\numberline}{\tablename~\oldnumberline}%
\listoftables%
}


\newpage


%\begin{spacing}{1}
%\begin{longtable}{|c|c|c|c|c|}
%    \caption{Title}
%    \label{tab-label-name} \\
%    \hline
%    \multicolumn{1}{|c|}{A}&
%    \multicolumn{1}{c|}{B}&
%    \multicolumn{1}{c|}{C}&
%    \multicolumn{1}{c|}{D}&
%    \multicolumn{1}{c|}{E}
%    \\
%    \hline
%    \endfirsthead
%    \multicolumn{5}{c}{{\tablename\ \thetable{} - continued}} \\
%    \hline
%    \multicolumn{1}{|c|}{A}&
%    \multicolumn{1}{c|}{B}&
%    \multicolumn{1}{c|}{C}&
%    \multicolumn{1}{c|}{D}&
%    \multicolumn{1}{c|}{E}
%    \\
%    \hline
%    \endhead
%    \hline
%    \endfoot
%    \hline
%    \endlastfoot
%    X
%    &X
%    &X
%    &X
%    &X
%    \\
%    \hline
%\end{longtable}
%\end{spacing}


\newpage 


\section*{Figures}
{%
\let\oldnumberline\numberline%
\renewcommand{\numberline}{\figurename~\oldnumberline}%
\listoffigures%
}


\newpage

%\begin{figure}[h!]
%    \centering
%    \includegraphics[width=0.X\textwidth]{.png}
%    \caption{}
%    \label{fig-label-name}
%\end{figure}

\begin{figure}[!ht]
    \centering
    \resizebox{\textwidth}{!}{
\begin{tikzpicture}
    %Pre-filter
    \draw[->] (-6,0)node[anchor=east]{$\dot{Q}_{HEX}$} -- (-4.5,0);
    \draw (-4.5,-0.5) rectangle (-3.5,0.5) node[pos=0.5]{$F(s)$};
    %Sum
    \draw[->] (-3.5,0) -- (-2,0) node[pos=0.5,anchor=south]{$\dot{Q}_{Core}^{SP}$};
    \draw (-1.75,0) circle (0.25) node{\scriptsize$\textbf{-}$};
    %Controller
    \draw[->] (-1.5,0) -- (-0.5,0)node[pos=0.5,anchor=south]{$e$};
    \draw (-0.5,-0.5) rectangle (0.5,0.5) node[pos=0.5]{$C(s)$};
    \draw[->] (0.5,0) -- (1.5,0) node[pos=0.5,anchor=south]{$u_{CD}$};
    %Actuator
    \draw (1.5,-0.5) rectangle (2.5,0.5) node[pos=0.5]{$A(s)$};
    \draw[->] (2.5,0) -- (3.5,0) node[pos=0.5,anchor=south]{$\rho_{CD}$};
    %Sum
    \draw (3.75,0) circle (0.25) node{\scriptsize$\textbf{+}$};
    %Process
    \draw[->] (4,0) -- (5,0) node[pos=0.5,anchor=south]{$\rho$};
    \draw (5,-0.5) rectangle (6,0.5) node[pos=0.5]{$P(s)$};
    \draw[->] (6,0) -- (8,0) node[anchor=west]{$\dot{Q}_{Core}$};
    %Transducer
    \draw[->] (7,0) -- (7,-1.5) -- (2.5,-1.5);
    \draw (1.5,-2) rectangle (2.5,-1) node[pos=0.5]{$H(s)$} ;
    \draw[->] (1.5,-1.5) -- (-1.75,-1.5) -- (-1.75,-0.25);
    %Passive Feedback
    %Core Feedback
    \draw[->] (7,0) -- (7,2.5);
    \draw (6.5,2.5) rectangle (7.5,3.5) node[pos=0.5]{$G_{C}(s)$};
    \draw[->] (6.5,3) -- (4.25,3);
    %HEX Feedback
    \draw[->] (-5,0) -- (-5,2.5);
    \draw (-4.5,2.5) rectangle (-5.5,3.5) node[pos=0.5]{$G_{H}(s)$};
    \draw[->] (-4.5,3) -- (3.25,3)node[pos=0.5,anchor=south]{$T_{cold}$};
    %Sum
    \draw (3.75,1.5) circle (0.25) node{\scriptsize$\textbf{+}$};
    \draw[->](3.75,1.25) -- (3.75,0.25);
    %TemperatureFeedback
    \draw (1.5,1) rectangle (2.5,2) node[pos=0.5]{$\alpha_T$};
    \draw[->] (2.5,1.5) -- (3.5,1.5) node[pos=0.5,anchor=south]{$\rho_{T}$};
    %FlowFeedback
    \draw (3.25,2.5) rectangle (4.25,3.5) node[pos=0.5]{$\alpha_F$};
    \draw[->] (3.75,2.5) -- (3.75,1.75) node[pos=0.5,anchor=west]{$\rho_{F}$};
    %Downcomer
    \draw[->] (0,3) -- (0,2);
    \draw (-0.5,1) rectangle (0.5,2) node[pos=0.5]{$\theta_{DC}$};
    \draw[->] (0.5,1.5) -- (1.5,1.5);
    %Riser
    \draw[->] (5.375,3) -- (5.375,4.5) -- (0.5,4.5) node[pos=0.5,anchor=south]{$T_{hot}$};
    \draw (-0.5,4) rectangle (0.5,5) node[pos=0.5]{$\theta_{R}$};
    \draw[->] (-0.5,4.5) -- (-5,4.5) -- (-5,3.5);


\end{tikzpicture}
}
    \caption[Control loop of a natural circulation \acs{msnb}]{Control loop of a natural circulation \acs{msnb}. It is a normal feedback loop with a pre-filter, with the addition of the passive feedback mechanisms. The core ($\dot{Q}_{Core}$) and heat exchanger ($\dot{Q}_{HEX}$) powers go through the respective temperature dynamics ($G_C$ and $G_H$) and time delays for the riser ($\theta_R$) and downcomer ($\theta_{DC}$) before being converted to reactivity by the temperature($\alpha_T$) and flow ($\alpha_F$) feedback mechanisms. The passive reactivity feedback is combined with the control drum reactivity ($\rho_{CD}$) and fed into the reactor dynamics ($P(s)$).  }
    \label{fig:ReactorControlLoop}
\end{figure}


\begin{tikzpicture}
    %blocks
    \draw (-2,-0.5) rectangle (-1,0.5) node[pos=0.5]{NC};
    \draw (1,-0.5) rectangle (2,0.5) node[pos=0.5]{RPK};
    \draw (-0.5,1.9) rectangle (0.5,2.9) node[pos=0.5]{USUF};
    
    %inner arrows
    \draw[->] (0,1.9) -- (0,0.85);
    \draw node at (0,0.7) {\tiny Temp-Profile};
    \draw[->] (-0.2,0.55) -- (-0.95,0);
    \draw[->] (0.2,0.55) -- (0.95,0);

    %outer arrows
    \draw[->] (-1.75,0.5) arc (180:112.5:2) node[midway,sloped,above]{\tiny Flow Rate};
    \draw[->] (-1.5,-0.5) arc (225:315:2) node[midway,sloped,below]{\tiny Flow Rate};
    \draw[->] (1.75,0.5) arc (0:67.5:2) node[midway,sloped,above]{\tiny Core Power};
\end{tikzpicture}

\newpage



%\begin{appendices}


%%%%% use for toc
%    \titleformat{\part}{\normalfont\Large\bfseries\filcenter}{\thepart}{0ex}{}[]
%    \part*{Appendices}
%%%%%


%    \setcounter{section}{0}
%    \renewcommand{\thesection}{\Roman{section}}
%    \titleformat{\section}{\normalfont\large\bfseries}{Appendix \thesection:}{1ex}{}[]


%    \section{Just appendix title} \label{app-}


%\end{appendices}


\end{document}